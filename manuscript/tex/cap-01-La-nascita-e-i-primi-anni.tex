\chapter{La nascita e i primi anni}

\narrator{Primo narratore.} La storia dell’India inizia con la vita del Buddha
Gotama. Più esattamente, è a questo punto che la memoria storica rimpiazza
l’archeologia e la leggenda. Le notizie fornite dalla vita e dagli insegnamenti
del Buddha – sono i primi testi a essere ritenuti storicamente attendibili –
indicano l’esistenza di una civiltà stabile e assai sviluppata che richiese
molto tempo per giungere a maturazione. Il Buddha ottenne l’Illuminazione nei
pressi di Uruvelā, nella pianura del Gange, chiamata la “Terra di Mezzo”. In
base al modo in cui sono misurate le distanze in India, non si trattava di
luoghi molto distanti dall’antichissima e santa città di Benares. Egli aveva
trentacinque anni, e per sei anni s’era sforzato per ottenere l’Illuminazione.
D’allora in poi, per quarantacinque anni, errò fra vari luoghi dell’India
centrale, spiegando in continuazione le Quattro Nobili Verità da lui scoperte.
Così come calcolato in Europa, il \emph{Parinibbāna} finale ebbe luogo nel 483
a.C., secondo la tradizione nel giorno di luna piena del mese di maggio. Il
periodo nel quale egli visse sembra essere stato eccezionalmente calmo, la
società era stabile e i governi ben organizzati, in forte contrasto rispetto a
quanto avvenne prima e dopo.

\narrator{Secondo narratore.} Tre mesi dopo il \emph{Parinibbāna} del Buddha, i
suoi discepoli più longevi sopravvissutigli convocarono un concilio di
cinquecento monaci anziani per mettersi d’accordo sulla forma in cui
l’insegnamento del Maestro doveva essere tramandato alla posterità. Tra questi
cinquecento monaci, che avevano tutti realizzato l’Illuminazione, Upāli fu
l’autorità riconosciuta per le regole di condotta e le circostanze che avevano
indotto a stilarle. La parte principale del “Canestro della Disciplina” – il
\emph{Vinaya Piṭaka} – fu composto durante il concilio, sulla base della sua
recitazione.

Quando ebbe finito, Ānanda fu invitato a recitare i Discorsi. Ānanda, che era
stato l’assistente personale del Buddha durante gli ultimi suoi ventiquattro
anni, era dotato di una memoria straordinaria. Quasi tutta la raccolta dei
Discorsi, con le rispettive ambientazioni, contenuti nel “Canestro dei Discorsi”
(il \emph{Sutta Piṭaka}) fu composto sulla base della sua recitazione. Upāli
iniziò ogni resoconto con le parole \emph{tena samayena} – «Avvenne questo» –
mentre Ānanda cominciò ogni discorso riferendo il luogo in cui il Buddha parlò e
la persona alla quale egli si rivolse, iniziando con le parole \emph{evaṃ me
  sutaṃ} – «Così ho udito».

\narrator{Primo narratore.} Questa narrazione della vita del Buddha è tratta da
tali due “Canestri”. Come siano riusciti a giungere fino a noi, lo diremo in
seguito. Qui, per cominciare, offriamo il racconto dell’ultima nascita del
Buddha, descritta da lui stesso e in seguito riferita da Ānanda durante il
concilio. Le parole furono pronunciate nella lingua del Buddha, nota come lingua
pāli.

\voice{Prima voce.} Così ho udito. Una volta il Beato\footnote{Una traduzione
  letterale dell’aggettivo \emph{bhagavant} è impossibile. Viene perciò reso con
  “Beato”. Buddhaghosa nel suo \emph{Visuddhimagga} (VII, pp. 53 ss.) offre
  numerose spiegazioni in merito.} soggiornava a Sāvatthī, nel Boschetto di
Jeta, nel parco di Anāthapiṇḍika. Un certo numero di bhikkhu\footnote{La parola
  “bhikkhu” (in sanscrito: \emph{bhikṣhu}) è stata lasciata in lingua pāli.
  Etimologicamente deriva da \emph{bhikkhā} (elemosina). Vi sono però altre
  derivazioni “semantiche” più antiche: \emph{saṃsāre bhayaṃ ikkhatī ti bhikkhu}
  («colui che vede la paura nel ciclo delle rinascite, e perciò è un “veggente
  della paura”»). Un bhikkhu è un membro a pieno titolo della comunità monastica
  (Saṅgha), ma il fatto che vi sia pienamente accolto non comporta alcun voto
  irrevocabile.} era in attesa nella sala delle riunioni, dove si erano raccolti
al ritorno dalla questua, dopo il termine del pasto. Frattanto, dicevano tra
loro: «È meraviglioso, amici, è stupefacente come il potere e l’energia del
Perfetto gli consentano di avere conoscenza dei Buddha del passato che
realizzarono la completa estinzione delle contaminazioni, che sbrogliarono la
matassa, che spezzarono il cerchio, che posero fine al girare in tondo e che
andarono al di là di ogni sofferenza, e così di sapere: queste furono le nascite
di quei Beati, questi i loro nomi, questa la loro stirpe, questa la loro virtù,
questa la loro concentrazione, questa la loro comprensione, questo il loro
dimorare, questo il modo della loro liberazione».

Dopo che queste parole furono pronunciate, il venerabile Ānanda disse ai
bhikkhu: «Gli Esseri Perfetti sono meravigliosi, amici, e hanno meravigliose
qualità. Gli Esseri Perfetti sono stupefacenti e hanno stupefacenti qualità».

Il loro discorso non era ancora terminato quando già era sera, e il Beato, che
aveva interrotto il suo ritiro, raggiunse la sala delle riunioni e si mise a
sedere nel posto preparatogli. Chiese allora ai bhikkhu: «Per parlare di quale
argomento vi siete qui riuniti? Quale discorso è stato lasciato incompiuto?».

Gli fu allora riferito quello che i bhikkhu e il venerabile Ānanda avevano
detto, e loro soggiunsero: «Signore, questo è il discorso lasciato incompiuto
perché è arrivato il Beato». Il Beato allora si rivolse al venerabile Ānanda:
«Se è così, Ānanda, spiega più a fondo le meravigliose e stupefacenti qualità
degli Esseri Perfetti».

«Ho sentito e imparato questo, Signore, dalle labbra stesse del Beato.
Consapevole e con piena presenza mentale il Bodhisatta, l’Essere Dedito
all’Illuminazione, apparve nel paradiso dei Gioiosi.\footnote{Il paradiso dei
  Gioiosi (\emph{Tusita}). La cosmologia di quei tempi descrive molti paradisi:
  in particolare sei paradisi nei quali sono goduti tutti i piaceri sensoriali;
  al di sopra di questi, dodici paradisi di Brahmā – il “Mondo della Suprema
  Divinità” – nei quali la consapevolezza è del tutto purificata dalla brama,
  benché non lo sia da una sua futura potenzialità. Secondo il Commentario, in
  questi ultimi paradisi la forma materiale è rarefatta dall’assenza dei tre
  sensi dell’odorato, del gusto e del tatto, nonché del sesso; tali paradisi
  corrispondono agli stati raggiungibili dagli esseri umani nei primi quattro
  jhāna (stati di assorbimento meditativo). Oltre a questi, per meglio dire,
  affinamenti dei quattro jhāna, vi sono quattro stati infiniti “privi di
  forma”, nei quali ogni percezione della forma materiale e delle differenze è
  trascesa: corrispondono all’infinità dello spazio e della coscienza, al
  nulla-è, alla né-percezione-né-non-percezione. La rinascita in ognuno di essi
  è impermanente e seguìta da nuove rinascite fino a quando non si ottiene il
  Nibbāna, il non-creato.} Ed è questo che io ricordo come meravigliosa e
stupefacente qualità del Beato».

«Ho sentito e imparato questo, Signore, dalle labbra stesse del Beato.
Consapevole e con piena presenza mentale il Bodhisatta restò nel paradiso dei
Gioiosi».

«Per l’intera durata di quella vita il Bodhisatta restò nel paradiso dei
Gioiosi».

«Consapevole e con piena presenza mentale scomparve dal paradiso dei Gioiosi e
discese nel grembo di sua madre».

«Quando il Bodhisatta scomparve dal paradiso dei Gioiosi ed entrò nel grembo di
sua madre, una luce grande e incommensurabile che superava per splendore quella
degli dèi apparve nel mondo con i suoi deva, con i suoi Māra e con le sue
divinità, in questa generazione con i suoi monaci e brāhmaṇa, con i suoi
principi e uomini.\footnote{Seguendo il Commentario, \emph{sadevamanussānaṃ} è
  stato tradotto con «con i suoi principi e uomini». È il senso complessivo a
  richiederlo, mentre “deva” era anche la forma normale per rivolgersi a un re.}
E anche nelle intercapedini di quell’abissale mondo fatto di vuoto, tenebre e
assoluta oscurità, dove la luna e il sole, potenti e possenti come sono, non
riescono a far prevalere la loro luce – pure là comparve una luce grande e
incommensurabile che superava per splendore quella degli dèi, e le creature nate
in queste intercapedini grazie a quella luce riuscirono a percepirsi
vicendevolmente: “Così, sembra che altre creature siano apparse qui!”. E questo
sistema di diecimila mondi si scosse, tremò e vacillò, e anche lì comparve una
luce grande e incommensurabile che superava per splendore quella degli dèi».

«Quando il Bodhisatta discese nel grembo di sua madre, quattro divinità giunsero
a proteggerlo dai quattro angoli del mondo, così che nessun essere umano,
non-umano né alcun altro potesse in alcun modo nuocere a lui o a sua madre».

«Quando il Bodhisatta discese nel grembo di sua madre, lei divenne
intrinsecamente pura, si astenne dall’uccidere esseri viventi, dal prendere quel
che non è dato, dal comportamento non casto, dalla falsa parola, e
dall’indulgere al vino, ai liquori e alle bevande fermentate».

«Quando il Bodhisatta discese nel grembo di sua madre, lei non fu più toccata
dai cinque aspetti del desiderio sensoriale, e divenne inaccessibile per gli
uomini di mente lussuriosa».

«Quando il Bodhisatta discese nel grembo di sua madre, lei nel contempo
possedeva i cinque aspetti del desiderio sensoriale, ed essendone dotata e
corredata, ne era gratificata».

«Quando il Bodhisatta discese nel grembo di sua madre, in lei non sorse alcun
genere di afflizione: era beata e priva di ogni affaticamento corporale».

«Come se un filo blu, giallo, rosso, bianco o marrone fosse introdotto in una
pregiata gemma di berillo ben tagliata a otto facce, trasparente come acqua
purissima, e un uomo con gli occhi sani la prendesse in mano esaminandola in
questo modo – “Questa è una pregiata gemma di berillo ben tagliata a otto facce,
trasparente come acqua purissima, e un filo blu, giallo, rosso, bianco o marrone
è introdotto in essa” – così anche la madre del Bodhisatta lo vedeva nel proprio
grembo con tutte le sue membra, dotato di ogni facoltà».

«Sette giorni dopo la nascita del Bodhisatta, sua madre morì e rinacque nel
paradiso dei Gioiosi».

«Altre donne partoriscono dopo aver tenuto in grembo il bambino per nove o dieci
mesi, ma non la madre del Bodhisatta. Lei partorì dopo averlo tenuto in grembo
per dieci mesi esatti».

«Altre donne partoriscono sedute o distese, ma non la madre del Bodhisatta. Lei
lo partorì stando in piedi».

«Quando il Bodhisatta uscì dal grembo della madre, egli non toccò la terra. Le
quattro divinità lo accolsero e lo posero di fronte alla madre, dicendo:
“Gioisci, o regina, hai dato alla luce un figlio con un grande potere”».

«Quando il Bodhisatta uscì dal grembo della madre, fu come se una gemma fosse
messa in un panno di Benares. La gemma non macchierebbe il panno né il panno
macchierebbe la gemma – perché no? – perché entrambi sono puri. Così anche il
Bodhisatta uscì dal grembo della madre immacolato, senza essere macchiato da
acqua, umori, sangue o qualsiasi altro genere d’impurità, uscì pulito e
immacolato.

«Quando il Bodhisatta uscì dal grembo della madre, due getti d’acqua si
riversarono dal cielo, uno fresco e uno caldo, per lavare il Bodhisatta e sua
madre».

«Appena il Bodhisatta nacque, rimase saldamente in piedi sul terreno, poi fece
sette passi a nord e, mentre un bianco parasole era tenuto sul suo capo,
sorvegliò ogni angolo del mondo. Proferì le parole del Signore del Branco: “Nel
mondo sono il Supremo, nel mondo sono il Migliore, nel mondo sono l’Eminente.
Questa è l’ultima nascita, ora non ci saranno più rinascite in vite future”».

«Quando il Bodhisatta uscì dal grembo della madre, una luce grande e
incommensurabile, che superava per splendore quella degli dèi, apparve nel mondo
con i suoi deva, con i suoi Māra e con le sue divinità, in questa generazione
con i suoi monaci e brāhmaṇa, con i suoi principi e uomini. E anche nelle
intercapedini di quell’abissale mondo fatto di vuoto, tenebre e assoluta
oscurità, dove la luna e il sole, potenti e possenti come sono, non riescono a
far prevalere la loro luce – pure là comparve una luce grande e incommensurabile
che superava per splendore quella degli dèi, e le creature nate in queste
intercapedini grazie a quella luce riuscirono a percepirsi vicendevolmente:
“Così, sembra che altre creature siano apparse qui!”. E questo sistema di
diecimila mondi si scosse, tremò e vacillò, e anche lì comparve una luce grande
e incommensurabile che superava per splendore quella degli dèi».

«Tutte queste cose ho udito e imparato dalle labbra stesse del Beato. E io le
ricordo come meravigliose e stupefacenti qualità del Beato».

«Se è così, Ānanda, ricorda anche questa come meravigliosa e stupefacente
qualità di un Essere Perfetto: le sensazioni piacevoli, dolorose o neutre di un
Essere Perfetto sono da lui conosciute quando sorgono, sono da lui conosciute
quando sono presenti, e sono da lui conosciute quando si placano; le sue
percezioni sono da lui conosciute quando sorgono, sono da lui conosciute quando
sono presenti, sono da lui conosciute quando si placano; i suoi pensieri sono da
lui conosciuti quando sorgono, sono da lui conosciuti quando sono presenti, sono
da lui conosciuti quando si placano».

«E anche questo ricordo, o Signore, come meravigliosa e stupefacente qualità del
Beato».

Questo è ciò che il venerabile Ānanda disse. Il Maestro approvò. I bhikkhu
furono soddisfatti, e si deliziarono delle parole del venerabile Ānanda.

\suttaRef{M. 123; cf. D. 14}

\narrator{Primo narratore.} Come un veggente brāhmaṇa – un veggente del “divino”
o della casta dei sacerdoti – predisse la futura Illuminazione è raccontato in
un canto.

\cantor{Cantore}

\begin{quote}
Il Saggio Asita, nella sua meditazione diurna, \\
vide che gli dèi, quelli della Compagnia dei Trenta, \\
erano felici e gioiosi e, vestiti di splendore, sventolavano bandiere, \\
rumorosamente si rallegravano assieme al loro sovrano Sakka. \\
Quando vide gli dèi così felici ed esultanti, \\
rispettosamente li salutò e rivolse loro questa domanda:

«Perché la Compagnia degli dèi è così gioiosa? \\
Perché sventolano bandiere in questo modo? \\
Mai ci fu una celebrazione del genere, \\
nemmeno dopo la battaglia con i dèmoni, \\
quando gli dèi vinsero e i dèmoni furono sconfitti. \\
Qual è la meraviglia che hanno udito e che tanto li delizia? \\
Guardate come cantano, gridano e strimpellano chitarre, \\
come applaudono e danzano ovunque. \\
O voi, che dimorate sugli ariosi picchi del Monte Meru, \\
vi prego, non lasciatemi nel dubbio, buoni signori».

«In una città dei Sakya, nella terra di Lumbinī \\
è nato nel mondo degli uomini \\
un Essere che otterrà l’Illuminazione, un Gioiello Inestimabile \\
che porterà loro benessere e floridezza. \\
Per questo siamo gioiosi in modo così stravagante. \\
L’Essere Unico, la Personalità Sublime, \\
il Signore di tutti gli Uomini e l’Eminente del genere umano \\
farà girare la Ruota nel Boschetto degli Antichi Veggenti \\
con il ruggito del leone, il sovrano di tutti gli animali».

Quando udì queste parole, il Saggio si affrettò, \\
andò nella dimora di Suddhodana. \\
Lì sedette: «Dov’è il bimbo?». \\
Ai Sakya chiese: «Mostratemelo» \\
Quando i Sakya mostrarono il bimbo ad Asita \\
il suo colore era puro \\
come i raggi d’oro brillante lavorato in un crogiolo, \\
splendente e chiaro.

La gioia del rapimento estatico inondò il cuore di Asita \\
nel vedere il bimbo luminoso come una fiamma e puro \\
come il Signore delle Stelle che cavalca nel cielo, \\
abbagliante come il sole in un autunno senza nubi \\
mentre nella volta celeste gli dèi tenevano sul suo capo \\
un parasole nervato da migliaia di cerchi, \\
brandendo dorati piumini scaccia-insetti, \\
senza che nessuno vedesse \\
chi reggeva il parasole e i piumini.

Il saggio dai capelli intrecciati, chiamato
Kaṇhasiri,\footnote{\emph{Kaṇhasiri} significa “Buio Splendore” (l’equivalente in sanscrito di \emph{Kaṇha} è \emph{Kṛṣṇa}).} \\
vedendo il bimbo come un gioiello d’oro su broccato, \\
con il bianco parasole tenuto sul suo capo, \\
lo accolse colmo di gioia e di felicità. \\
Appena ricevette il Signore dei Sakya, \\
l’esperto interprete di marchi e segni \\
esclamò con cuore pronto e fiducioso: \\
«Tra la razza dei bipedi egli è unico». \\
Ricordò e vedendo il suo stesso destino \\
per la grande tristezza le lacrime gli velarono gli occhi. \\
I Sakya lo videro piangere, e gli chiesero: \\
«Qualche sventura accadrà al nostro principe?». \\
Ai Sakya ansiosi egli rispose: \\
«Prevedo che nessun pericolo toccherà il bimbo, \\
tanto meno qualche rischio l’attende. \\
Siate certi che non è uomo di secondo rango, \\
perché egli raggiungerà la sommità della vera conoscenza. \\
Un profeta d’impareggiabile purezza, \\
grazie alla compassione per la moltitudine metterà \\
in moto la Ruota del Dhamma e diffonderà la sua santa vita. \\
A me resta poco però da vivere, \\
nel frattempo morirò. Non potrò ascoltare \\
l’incomparabile Eroe insegnare il Buon Dhamma. \\
È questo a intristirmi, è questa la perdita che m’addolora».

Colui che visse la santa vita lasciò la stanza centrale del palazzo \\
dopo aver colmato i Sakya di sovrabbondante gioia. \\
Andò dal figlio di sua sorella e, mosso da compassione, \\
gli disse del futuro dell’impareggiabile Eroe che trova il Dhamma.

«Quando sarai raggiunto dalla notizia che egli è illuminato, \\
e sta vivendo il Dhamma da lui stesso scoperto, \\
va da lui, chiedigli il suo insegnamento \\
e vivi con il Beato la santa vita».

Così Nālaka, che aveva accumulato grandi meriti, \\
avvisato da chi il suo bene voleva, da chi aveva predetto \\
la venuta dell’Essere Perfetto, conseguì la massima purezza, \\
attese con i sensi raffrenati, aspettando il Vittorioso.

Sentendo che il Nobile Vittorioso \\
aveva fatto girare la Ruota, andò da lui. \\
Vide il Signore di tutti i Veggenti, \\
e credette in lui quando lo vide. \\
Adempiendo il volere di Asita, \\
egli chiese al Perfetto Saggio \\
del Silenzio Supremo.
\end{quote}

\suttaRef{Sn. 3:11}

\narrator{Primo narratore.} Benché la letteratura successiva offra molti
dettagli sui primi anni, il \emph{Tipiṭaka} dice pochissimo in proposito. Fa
riferimento a due soli episodi. Innanzitutto, il ricordo della meditazione sotto
l’albero di melarosa mentre il padre del Bodhisatta era al lavoro. Stava
svolgendo l’aratura cerimoniale per l’apertura della stagione della semina, dice
il Commentario. È un ricordo sul quale ci soffermeremo in seguito. Poi il
racconto delle “tre riflessioni”, che corrispondono a tre dei “messaggeri” –
l’anziano, il malato e il defunto – visti dal precedente Buddha Vipassī.

\suttaRef{D. 14}

\voice{Prima voce.} «Ero delicato, molto delicato, massimamente
delicato.\footnote{Queste circostanze sono altrove considerate essere costanti
  per tutti i Bodhisatta nella loro ultima esistenza (D. 14). Nel
  \emph{Tipiṭaka} la narrazione dei “quattro messaggeri” – l’anziano, il malato,
  il cadavere e il monaco – è riferita solo al precedente Buddha Vipassī, non al
  Buddha Gotama. Racconti successivi la collegano anche al Buddha Gotama.}
Laghetti adorni di fiori erano allestiti nella casa di mio padre per mio solo
beneficio. In uno fiorivano gigli blu, in un altro gigli bianchi, in un altro
ancora gigli rossi. Non usavo legno di sandalo a meno che non provenisse da
Benares. Il mio turbante, la mia tunica, gli indumenti della parte più bassa del
corpo e il mantello erano fatti di stoffa di Benares. Un bianco parasole era
tenuto sul mio capo di giorno e di notte, così che né il freddo né il caldo, né
la polvere né la sabbia e neanche la rugiada potessero infastidirmi».

«Avevo tre palazzi. Uno per la stagione fredda, uno per la stagione calda e un
altro per quella delle piogge. Nel palazzo per le piogge ero intrattenuto da
menestrelli, tra i quali non c’erano uomini. Durante i quattro mesi delle piogge
non scendevo mai nella parte inferiore del palazzo. Benché in altre case i pasti
per i domestici e gli inservienti prevedevano riso spezzato e zuppa di
lenticchie, nella casa di mio padre a loro era dato riso bianco e carne».

«Mentre godevo di tale autorità e buona sorte, tuttavia pensavo: “Quando un uomo
ignorante e ordinario, che è soggetto all’invecchiamento, non è al sicuro
dall’invecchiamento, vede un altro che è anziano, si sente scosso, umiliato e
disgustato perché dimentica che lui stesso non fa eccezione. Anch’io sono
soggetto all’invecchiamento, non sono al sicuro dall’invecchiamento, e perciò
non mi si addice essere scosso, umiliato e disgustato vedendo un altro che è
anziano”. Quando facevo questa riflessione, la vanità della giovinezza mi
abbandonava del tutto».

«Pensavo: “Quando un uomo ignorante e ordinario, che è soggetto alle malattie,
non è al sicuro dalle malattie, vede un altro che è malato, si sente scosso,
umiliato e disgustato perché dimentica che lui stesso non fa eccezione. Anch’io
sono soggetto alle malattie, non sono al sicuro dalle malattie, e perciò non mi
si addice essere scosso, umiliato e disgustato vedendo un altro che è malato”.
Quando facevo questa riflessione, la vanità della salute mi abbandonava del
tutto».

«Pensavo: “Quando un uomo ignorante e ordinario, che è soggetto alla morte, non
è al sicuro dalla morte, vede un altro che è morto, si sente scosso, umiliato e
disgustato perché dimentica che lui stesso non fa eccezione. Anch’io sono
soggetto alla morte, non sono al sicuro dalla morte, e perciò non mi si addice
essere scosso, umiliato e disgustato vedendo un altro che è morto”. Quando
facevo questa riflessione, la vanità della vita mi abbandonava del tutto».

\suttaRef{A. 3:38}

