\chapter{La persona}

\narrator{Primo narratore.} Per un periodo non è più possibile tracciare il
corso degli eventi, e ora bisogna fare una pausa per vedere che cosa dice il
Canone sulle qualità personali del Buddha: per vedere sia che cosa il Buddha
avesse da dire su se stesso sia che cosa disse di lui la gente che allora lo
incontrò, nelle più antiche relazioni che sono giunte fino a noi.

\voice{Prima voce.} Così ho udito. Una volta, quando il Beato soggiornava a
Sāvatthī nel Boschetto di Jeta, nel parco di Anāthapiṇḍika, si mise a sedere
passando in rassegna le molte cose non salutari che aveva abbandonato e le molte
cose salutari perfezionate sviluppandole dentro se stesso. Conoscendo il
significato di ciò, egli esclamò queste parole:

\begin{quote}
Quel che prima era, poi non fu. \\
Quel che prima non era, poi fu. \\
Quel che né non è stato né non sarà \\
ora non è.\footnote{La prima riga di questo enigma si riferisce alle contaminazioni della brama, dell’odio e dell’illusione, la seconda alla virtù, la terza e la quarta al momento dell’Illuminazione. Così il Commentario.}
\end{quote}

\suttaRef{Ud. 6:3}

Il Beato si mise di nuovo a sedere passando in rassegna l’abbandono della
varietà della proliferazione [mentale]\pagenote{%
  \emph{Papañca}. Per una differente interpretazione di questo difficile termine
  (NDT: reso da Bhikkhu Ñāṇamoli con “diversification”, “diversifying”), si veda
  \emph{Concept and Reality in Early Buddhist Thought} by BHIKKHU ÑĀṆANANDA
  (Kandy, BPS, 1971), dove, alla p. 21, è offerta una traduzione alternativa del
  verso seguente e della sua spiegazione nei Commentari (Nyp.).}
in se stesso. Conoscendo il significato di ciò, esclamò queste parole:

\begin{quote}
Chi, senza alcuna base per la proliferazione [mentale], \\
ha messo da parte il vincolo e anche l’ostruzione, \\
e vive come un veggente libero dalla brama, \\
dal mondo con le sue divinità non è sdegnato.
\end{quote}

\suttaRef{Ud. 7:7}

«Bhikkhu, ci sono queste Quattro Nobili Verità: la Nobile Verità della
Sofferenza, la Nobile Verità dell’Origine della Sofferenza, la Nobile Verità
della Cessazione della Sofferenza, e la Nobile Verità del Sentiero che conduce
alla Cessazione della Sofferenza. Un Essere Perfetto, realizzato e completamente
illuminato, è chiamato in questo modo perché ha scoperto queste Quattro Nobili
Verità così come veramente sono».

\suttaRef{S. 56:23}

\narrator{Secondo narratore.} Il Buddha nomina i sei Buddha che lo hanno preceduto.

\voice{Prima voce.} «Novantuno ere fa, bhikkhu, il beato Vipassī, realizzato e
completamente illuminato, apparve nel mondo. Trentuno ere fa il beato Sikhī,
realizzato e completamente illuminato, apparve nel mondo. Nella stessa
trentunesima era il beato Vessabhū, realizzato e completamente illuminato,
apparve nel mondo. In questa era fausta il beato Kakusandha, realizzato e
completamente illuminato, apparve nel mondo. In questa stessa era fausta, il
beato Koṇāgamana, realizzato e completamente illuminato, apparve nel mondo. In
questa stessa era fausta, il beato Kassapa, realizzato e completamente
illuminato, apparve nel mondo. In questa stessa era fausta, ora, io, realizzato
e completamente illuminato, sono apparso nel mondo».

\suttaRef{D. 14 (condensato)}

\narrator{Primo narratore.} Dopo aver descritto gli altri, ecco che cosa dice di
se stesso.

\voice{Prima voce.} «Io appartengo ai Khattiya, nobile stirpe guerriera. Sono
rinato in una famiglia khattiya. Quanto al mio clan sono un Gotama. La mia vita
è corta, di breve durata e termina presto; chi vive a lungo ora compie un secolo
o poco più. Ottenni l’Illuminazione ai piedi di un baniano \emph{assatha}, il
mio albero dell’Illuminazione. I miei discepoli eminenti sono Sāriputta e
Moggallāna. Ho un’assemblea composta di mille duecento cinquanta dicepoli, tutti
Arahant. Il mio monaco attendente, il mio principale attendente, è il bhikkhu
Ānanda. Un re, di nome Suddhodana, fu mio padre. Una regina, di nome Māyā, fu la
madre che mi generò. La città capitale del regno era Kapilavatthu».

\suttaRef{D. 14 (condensato)}

Questo fu detto dal Beato, fu detto dal Realizzato, così ho udito: «Bhikkhu, il
mondo è stato scoperto dal Perfetto, il Tathāgata: il Perfetto si è dissociato
dal mondo. L’origine del mondo è stata scoperta dal Perfetto: il Perfetto ha
abbandonato l’origine del mondo. La cessazione del mondo è stata scoperta dal
Perfetto: il Perfetto ha realizzato la cessazione del mondo. Il Sentiero che
conduce alla cessazione del mondo è stato scoperto dal Perfetto: il Perfetto ha
mantenuto in essere il Sentiero che conduce alla cessazione del mondo».

\margintodo{missing closing »}%
«Nel mondo con le sue divinità … qualsiasi cosa possa essere vista, udita,
percepita (per mezzo del naso, della lingua o del corpo), e conosciuta, o
raggiunta, ricercata e abbracciata con la mente, è stata scoperta dal Perfetto:
per questa ragione Egli è chiamato Perfetto (Tathāgata). Tutto quel che dice,
tutto quel che esclama, nel periodo compreso tra la notte in cui Egli scoprì la
suprema e totale Illuminazione e la notte in cui otterrà il Nibbāna definitivo,
l’elemento del Nibbāna senza alcun residuo del precedente attaccamento, è vero
(\emph{tathā}), nient’altro che vero. Per questa ragione egli è chiamato
Perfetto (Tathāgata). Come egli dice, così (\emph{tathā}) egli fa. Come egli fa,
così (\emph{tathā}) egli dice. Per questa ragione egli è chiamato Perfetto
(Tathāgata). Nel mondo con le sue divinità … il Perfetto è colui che è l’Essere
Trascendente Intrasceso, Veggente del Tutto e Detentore dei Poteri. Per questa
ragione egli è chiamato Perfetto (Tathāgata).

\suttaRef{Iti. 112; A. 4:23}

«Qualsiasi cosa in questo mondo con le sue divinità … sia vista, udita,
percepita, e conosciuta, o raggiunta, ricercata e abbracciata con la mente io la
conosco, io l’ho direttamente conosciuta. Benché ciò sia ammesso dal Perfetto,
non di meno egli non lo usa come una base (per la presunzione). Se dovessi dire
di tutto ciò che “io non lo conosco”, parlerei falsamente. E se dovessi dire di
esso che “io lo conosco e non lo conosco”, parlerei falsamente pure in questo
caso. E se dovessi dire di esso che “io né lo conosco né non lo conosco”, ciò
sarebbe scorretto da parte mia. Così, avendo visto quel che può essere visto, un
Essere Perfetto non concepisce la presunzione\pagenote{%
  Al verbo \emph{maññati} (“concepire la presunzione”) nei sutta corrispondono i
  sostantivi \emph{maññanā} (concezione) e \emph{māna} (presunzione, orgoglio).
  Utilizzato nel senso di concepire che “questo è quello” o semplicemente che
  “esso è”, esso ha un significato ontologico fondamentale (cf. M. 1 e M. 49)
  nell’attribuzione dell’“esistenza” a ciò che è percepito. Per il suo
  significato di “concepisco io sono” (\emph{asmi-māna}), si veda il cap.~12,
  pag.~\pageref{pag259} -- \emph{Com’è che perviene a esistere
    l’opinione\ldots}. Concependo che “io sono meglio di un altro”, ecc., si
  concepisce con orgoglio (\emph{atimāna}). È importante preservare questo filo
  di significati nei sutta.}
per ciò che è visto, non concepisce la presunzione per ciò che non è visto, non
concepisce la presunzione per ciò che potrebbe essere visto, non concepisce la
presunzione di alcun veggente. Avendo sentito quel che può essere udito … Avendo
percepito con i sensi quel che può essere percepito … Avendo conosciuto quel che
può essere conosciuto … egli non concepisce la presunzione di alcun conoscitore.
Un Essere Perfetto talmente equanime verso le cose viste, sentite, percepite con
i sensi o conosciute, rimane così equanime, e non c’è alcun’altra equanimità che
sia al di là o superiore a quella equanimità, così dico».

\suttaRef{A. 4:24}

Il re Pasenadi di Kosala chiese al Beato: «Signore, questo ho udito: “Il monaco
Gotama dice: ‘Non c’è monaco o brāhmaṇa che possa pretendere di avere la
completa conoscenza e visione di chi è onnisciente e tutto vede: questo non è
possibile’”. Signore, coloro che lo affermano, forse dicono quel che è stato
detto dal Beato e non travisano il Beato con ciò che di fatto non è, ed
esprimono idee in accordo con il Dhamma, senza che nulla sia legittimamente
deducibile dalle loro affermazioni che offra ragioni per condannarli?».

«Gran re, coloro che affermano di dire quel che non è stato detto da me mi
travisano».

«Allora, Signore, potrebbe essere stato detto in riferimento a qualcos’altro che
il Beato disse, e la persona credette che fosse altrimenti? Ad ogni modo,
Signore, in che modo il Beato conosce l’enunciazione che è stata pronunciata?».

«Io conosco un’enunciazione che è stata pronunciata in questo modo, gran re:
“Non c’è monaco o brāhmaṇa che conosce tutto, vede tutto, in un solo momento”».

«Quel che il Beato dice è ragionevole».

\suttaRef{M. 90}

\label{pag206}%
«Un Perfetto ha questi dieci poteri di un Perfetto, possedendo i quali egli
rivendica il ruolo di guida del gregge, con il ruggito del leone presiede le
assemblee e mette in moto l’incomparabile Ruota di Brahmā. Quali dieci?».

«Un Perfetto comprende, così come in realtà è, il possibile come possibile e
l’impossibile come impossibile».

«Egli comprende, così come in realtà è, con le connesse possibilità e ragioni,
la passata, la futura e la presente tendenza alla maturazione delle azioni
compiute».

«Egli comprende nella stessa maniera dove conducono tutte le strade».

«Egli comprende nella stessa maniera il mondo con i suoi molti e vari elementi».

«Egli comprende nella stessa maniera le differenti inclinazioni degli esseri».

«Egli comprende nella stessa maniera le disposizioni delle facoltà spirituali
negli altri esseri, nelle altre persone».

«Egli comprende nella stessa maniera la corruzione, la purificazione e il
progressivo emergere nelle meditazioni, nelle liberazioni, nelle concentrazioni
e nelle realizzazioni».

«Egli rammemora la sua multiforme vita passata …».

«Con l’occhio divino, che è purificato e supera quello umano, egli vede gli
esseri scomparire e ricomparire … egli comprende come gli esseri muoiano in
accordo con le loro azioni».

«Mediante la realizzazione di se stesso con la conoscenza diretta, egli qui e
ora entra e dimora nella liberazione della mente e nella liberazione mediante
comprensione immacolata per l’esaurimento delle contaminazioni».

\suttaRef{M. 12; cf. A. 10:21}

«Un Perfetto ha questi quattro generi di audacia,\footnote{Oppure perfetta
  sicurezza di sé, fiducia (\emph{vesārajja}) (Nyp.).} possedendo i quali egli
rivendica il ruolo di guida del gregge … : …».

«Non scorgo alcun indizio per cui nel mondo un monaco o un brāhmaṇa o una
divinità, o Māra o Brahmā, possa a ragione accusarmi in questo modo: “Tu, che
pretendi di essere completamente illuminato, non hai ancora scoperto queste
cose”. Oppure: “In te, che pretendi di aver esaurito le contaminazioni, queste
contaminazioni non sono ancora esaurite”. Oppure: “Queste cose che tu hai detto
essere delle ostruzioni, in realtà non sono ostruzioni per chi le pratica”.
Oppure: “Quando il tuo Dhamma è insegnato a beneficio di qualcuno, esso non
conduce alla completa estinzione della sofferenza in chi lo pratica”. Non
scorgendo indizi in tal senso, dimoro sicuro, privo di preoccupazioni e timori».

\suttaRef{M. 12}

Questo fu detto dal Beato, dal Realizzato, così ho udito:

«Due pensieri spesso sorgono in un Perfetto, realizzato e completamente
illuminato: il pensiero dell’innocuità e il pensiero della solitudine. Un
Perfetto prova piacere e si delizia nella non-afflizione, e con ciò spesso
pensa: “Con questo comportamento non affliggo nessuno, timido o spavaldo”. Un
Perfetto prova piacere e si delizia nella solitudine, e con ciò spesso pensa:
“Quel che è non salutare è stato abbandonato”».

\suttaRef{Iti. 38}

«Bhikkhu, non abbiate timore dei meriti. Meriti significa piacere, ciò che si
cerca e si desidera, che è piacevole e si ama. Ho avuto conoscenza diretta
mediante esperienza per un lungo periodo di ciò che si cerca e si desidera, che
è piacevole e si ama in quanto maturazione dei meriti di un lungo periodo. Dopo
aver mantenuto in essere la meditazione della gentilezza amorevole per sette
anni, non sono tornato in questo mondo per sette ere di contrazione e di
espansione del mondo. Nell’era in cui il mondo si stava contraendo sono andato
nel paradiso dei Brahmā della Fluente Radiosità. Nell’era in cui il mondo si
stava espandendo sono rinato nella vacua dimora di Brahmā. Là io fui un Brahmā,
un Gran Brahmā, un Essere Trascendente Intrasceso, un Veggente del Tutto, un
Detentore dei Poteri. Sono stato trentasei volte Sakka, un Sovrano degli dèi
(della sensorialità). Sono stato molte centinaia di volte un re come retto
Monarca Universale che gira la ruota, vittorioso in tutti e quattro i punti
cardinali, con il mio regno stabile e in possesso dei sette tesori. Che cosa è
necessario dire della sovranità mondana? Pensai: “Di quale mia azione questo è
il frutto, la maturazione del fatto che sono così possente e poderoso?”. Allora
mi venne da pensare: “È il frutto, la maturazione di tre tipi di mie azioni il
fatto che sono così possente e poderoso, ossia del donare, dell’[auto]controllo
e del contenimento”».

\suttaRef{Iti. 22}

Una volta il Beato stava viaggiando sulla strada tra Ukkaṭṭhā e Setavyā, e anche
il brāhmaṇa Doṇa stava viaggiando su quella strada. Egli vide nelle orme del
Beato delle ruote con mille raggi, con cerchi e mozzi al completo. Allora pensò:
«È meraviglioso, è magnifico! Certo queste non possono essere le orme di un
essere umano».

Allora il Beato lasciò la strada e si mise a sedere ai piedi di un albero, a
gambe incrociate, con il corpo eretto e con la consapevolezza fissa davanti a
lui. Allora il brāhmaṇa Doṇa, che stava seguendo le impronte, lo vide seduto ai
piedi dell’albero. Il Beato ispirava fiducia e sicurezza, le sue facoltà erano
rasserenate, la sua mente era quieta e aveva raggiunto il supremo controllo e la
suprema serenità: un pachiderma autocontrollato e custodito dal contenimento
delle facoltà sensoriali. Il brāhmaṇa andò da lui e gli chiese: «Signore, sarai
un dio?».

«No, brāhmaṇa».

«Signore, sarai un angelo celeste?».

«No, brāhmaṇa».

«Signore, sarai uno spirito?».

«No, brāhmaṇa».

«Signore, sarai un essere umano?».

«No, brāhmaṇa».

«Signore, che cosa invero sarai allora?».

«Brāhmaṇa, le contaminazioni per mezzo delle quali, non avendole abbandonate,
potrei essere un dio, un angelo celeste, uno spirito o un essere umano sono
state da me abbandonate, recise alla radice, rese come un ceppo di palma,
eliminate, e non sono più soggette e sorgere in futuro. Proprio come un loto
blu, rosso o bianco nasce nell’acqua, cresce nell’acqua e spunta dall’acqua
senza essere da essa toccato, così anch’io, che sono nato nel mondo e cresciuto
nel mondo, ho trasceso il mondo e vivo senza essere toccato dal mondo. Ricordami
come un Illuminato».

\suttaRef{A. 4:36}

Una volta il Beato stava di nuovo viaggiando nel territorio dei Videha con un
largo seguito di bhikkhu, con cinquecento bhikkhu. Ora, in quel tempo il
brāhmaṇa Brahmāyu viveva a Mithilā. Era vecchio, anziano, appesantito dagli
anni, avanti nella vita e giunto allo stadio finale. Si trovava nel suo
centoventesimo anno. Era esperto nei tre Veda, conosceva il testo e il contesto
degli \emph{Itihāsa}, la quinta delle autorità brahmaniche con le loro
invocazioni, liturgie e analisi terminologiche, ed era del tutto versato nella
scienza naturale e in quella dei segni del Grande Uomo.

Egli aveva sentito parlare delle qualità del Beato e del fatto che stava
viaggiando nel territorio dei Videha. Aveva un discepolo, un giovane studente
brāhmaṇa di nome Uttara, che era tanto esperto quanto il suo maestro e
altrettanto versato nella scienza dei segni del Grande Uomo. Il brāhmaṇa disse
al suo discepolo: «Vieni, mio caro Uttara, va dal monaco Gotama e scopri se la
fama che su di lui si è ovunque diffusa è vera o no, e se egli è uno così oppure
no. Per mezzo di te noi vedremo il monaco Gotama».

«Come farò a trovarlo, però, signore?».

«Mio caro Uttara, i trentadue segni del Grande Uomo sono stati registrati nelle
nostre scritture, e il Grande Uomo che ne è dotato ha solo due possibili
destini, non altri. Se vive la vita famigliare, egli diviene un retto Monarca
Universale, un conquistatore dei quattro angoli del mondo, un invitto, che rende
stabile il suo regno e possiede i sette tesori: il tesoro della ruota, il tesoro
dell’elefante, il tesoro del cavallo, il tesoro dei gioielli, il tesoro della
donna, il tesoro del capofamiglia e, come settimo, il tesoro del consigliere. I
suoi figli, che superano il numero di mille, sono coraggiosi, eroici e
annientano gli eserciti nemici. Su questa terra, lambita dall’oceano, egli
governa senza un bastone, senza un’arma e con rettitudine. Se però abbandona la
vita famigliare per la vita religiosa, egli diventa un Realizzato, un
completamente illuminato, che allontana il velo del mondo. Io, però, mio caro
Uttara, sono colui che ti ha passato le scritture, tu sei colui che le ha
ricevute».

«E sia, signore», egli rispose.

Egli si alzò dal posto in cui sedeva e, dopo aver prestato omaggio al brāhmaṇa,
se ne andò girandogli a destra verso il luogo in cui il Beato errava nel
territorio dei Videha. Viaggiando per tappe, giunse nel luogo in cui il Beato si
trovava. Scambiò dei saluti con lui e, quando questi formali doveri di cortesia
ebbero termine, si mise a sedere da un lato. Dopo averlo fatto, cercò i
trentadue segni del Grande Uomo sul corpo del Beato. Egli vide, più o meno, i
trentadue segni, eccetto due. Era dubbioso e incerto su due dei segni e non
riusciva a prendere una decisione e a convincersi in relazione a essi, a
riguardo di quel che, celato nella veste, avrebbe dovuto essere racchiuso nel
prepuzio e a riguardo della grandezza della lingua.

Allora al Beato venne in mente che egli era in dubbio in relazione a tali due
segni. Operò allora un atto di potere soprannaturale, così che il discepolo
brāhmaṇa Uttara vide che nel Beato quel che era celato nella veste era racchiuso
nel prepuzio. Allora il Beato estrasse la lingua e toccò ripetutamente entrambi
i fori degli orecchi, toccò ripetutamente entrambi i fori delle narici e coprì
tutta la fronte con la lingua. Allora il brāhmaṇa pensò: «Il monaco Gotama è
dotato dei trentadue segni del Grande Uomo. E se io lo seguissi e osservassi
come si comporta?».

Allora egli lo seguì per sette mesi come un’ombra, senza mai lasciarlo. Alla
fine dei sette mesi partì dal territorio dei Videha per tornare a Mithilā.

Andò da Brahmāyu il brāhmaṇa, gli prestò omaggio e si mise a sedere da un lato.
Allora il brāhmaṇa gli chiese: «Bene, mio caro Uttara, la fama che sul monaco
Gotama si è diffusa è vera o no? E il Maestro Gotama è uno così oppure no?».

«La fama è vera, signore, non falsa. Il Maestro Gotama è uno così, non altro.
Ora, il Maestro Gotama poggia i piedi in terra ad angolo retto, questo è in lui
il segno del Grande Uomo. Sulle piante dei suoi piedi ci sono ruote con mille
raggi, con cerchi e mozzi al completo … Egli ha calcagni sporgenti … Egli ha
lunghe dita delle mani e dei piedi … Le sue mani e i suoi piedi sono soffici e
gentili … Egli ha belle mani … I suoi piedi sono arcuati … Le sue gambe sono
come quelle di un’antilope … Quando sta in piedi, senza chinarsi entrambe le
palme delle sue mani toccano e strofinano entrambe le sue ginocchia … Quel che
di lui è celato nella veste è racchiuso nel prepuzio … Egli ha il colore
dell’oro … La sua pelle ha lucentezza dorata, ma è sottile e, a causa della
sottigliezza della sua pelle, la polvere e la sporcizia non si attaccano al suo
corpo … I peli del suo corpo crescono singolarmente, ogni pelo cresce da solo
nel suo poro … Le estremità dei peli del suo corpo si volgono verso l’alto, e
sono di colore nero bluastro, lucidi, ricci e piegati a destra … Egli ha gli
arti dritti di un Brahmā … Egli ha sette convessità … La parte superiore del suo
tronco è quella di un leone … Il solco tra le sue spalle è piatto … Egli ha le
proporzioni di un baniano, l’ampiezza delle sue braccia eguaglia l’altezza del
suo corpo, e l’altezza del suo corpo eguaglia l’ampiezza delle sue braccia … Il
suo collo e le sue spalle sono allineate … Il suo senso del gusto è estremamente
acuto … Egli ha le mascelle di un leone … Egli ha quaranta denti … I suoi denti
sono regolari … Non c’è spazio tra un dente e l’altro … I suoi denti sono
bianchissimi … Egli ha una grande lingua … Egli ha una voce divina, come quella
di un uccello Keravīka … I suoi occhi sono molto neri … Egli ha le ciglia di un
bue … Nello spazio tra le sue sopracciglia crescono [peli] bianchi, lucenti come
soffice cotone … Il suo capo ha la forma di un turbante, anche questo è un segno
in lui del Grande Uomo. Così, il Maestro Gotama è dotato di questi trentadue
segni del Grande Uomo».

«Quando cammina, comincia a farlo con il piede destro. Egli non poggia il piede
né troppo lontano né troppo vicino. Egli non cammina né troppo veloce né troppo
lento. Egli cammina senza che le sue ginocchia si tocchino. Egli cammina senza
che le sue caviglie si tocchino. Egli cammina senza alzare o abbassare le cosce,
né avvicinarle l’una all’altra né discostarle. Quando egli cammina, solo la
parte inferiore del suo corpo oscilla, ed egli cammina senza alcuno sforzo
corporeo. Quando egli si volta per guardare, lo fa con tutto il suo corpo. Egli
non guarda verticalmente verso il basso. Egli non guarda verticalmente verso
l’alto. Egli non cammina guardandosi attorno. Egli guarda davanti a sé per
l’ampiezza di un giogo d’aratro ma, al di là di questo, ha la visione di una
conoscenza priva d’impedimento».

«Quando entra in una dimora, non alza né abbassa il suo corpo, e neanche lo
curva in avanti o indietro. Egli si volta quando non è troppo lontano né è
troppo vicino al luogo in cui siede. Egli non si sporge con le mani verso il
luogo in cui siede. Egli non proietta in giù il suo corpo verso il luogo in cui
siede».

«Quando è seduto all’interno, non agita le mani. Egli non agita i piedi. Egli
non siede a ginocchia incrociate. Egli non siede a caviglie incrociate. Egli non
siede con la mano che regge il mento. Quando è seduto all’interno, egli non ha
timore, egli non rabbrividisce né trema, egli non è nervoso. Non gli si rizzano
i capelli per questo motivo, ed è intento all’isolamento».

«Quando egli riceve acqua per la ciotola, non alza né abbassa la ciotola, né la
inclina in avanti o indietro. Egli non riceve né poca acqua né troppa acqua
nella ciotola. Egli lava la ciotola senza sciacquettii. Egli lava la ciotola
senza capovolgerla. Egli non poggia la ciotola in terra per lavarsi le mani,
quando le sue mani sono lavate la ciotola è lavata e quando la ciotola è lavata
le sue mani sono lavate. Per gettare via l’acqua dalla ciotola, egli la versa
non troppo lontano né troppo vicino, né la versa sopra [qualcosa]».

«Quando egli riceve il riso, non alza né abbassa la ciotola, né la inclina in
avanti o indietro. Egli non riceve né poco riso né troppo riso. Egli aggiunge
salse nella giusta proporzione, non esagera la giusta quantità di salsa per un
boccone. Egli sposta il boccone per masticarlo spostandolo da una parte
all’altra della sua bocca e poi lo deglutisce, e non c’è grano di riso che entri
nel suo corpo senza essere stato masticato né che rimanga nella sua bocca, poi
prende un altro boccone. Egli assume il suo cibo sperimentando il sapore senza
sperimentare avidità per il sapore. Il cibo che egli assume ha cinque fattori:
non è per svago né per ebbrezza né per abbellirsi, ma solo per far durare e far
continuare a vivere questo corpo, per porre termine al disagio e per sussidio
alla santa vita: “In questo modo esaurirò le vecchie sensazioni senza farne
sorgere di nuove, e vivrò irreprensibile con agio e salute”».

«Quando egli ha mangiato e riceve acqua per la ciotola, non alza né abbassa la
ciotola, né la inclina in avanti o indietro. Egli non riceve né poca acqua né
troppa acqua nella ciotola. Egli lava la ciotola senza sciacquettii. Egli lava
la ciotola senza capovolgerla. Egli non poggia la ciotola in terra per lavarsi
le mani, quando le sue mani sono lavate la ciotola è lavata e quando la ciotola
è lavata le sue mani sono lavate. Per gettare via l’acqua dalla ciotola, egli la
versa non troppo lontano né troppo vicino, né la versa sopra [qualcosa]».

«Quando ha mangiato, egli poggia la ciotola in terra non troppo lontana né
troppo vicina, e non è né trascurato né troppo sollecito in relazione a essa».

«Quando ha mangiato, egli siede in silenzio per un po’, ma non lascia che il
tempo per la benedizione venga meno. Quando impartisce la benedizione dopo aver
mangiato, non lo fa criticando il pasto o attendendosene un altro. Egli
istruisce, esorta, risveglia e incoraggia l’uditorio con soli discorsi di
Dhamma. Quando ha terminato di farlo, si alza dal posto in cui siede e si
allontana».

«Egli cammina non troppo veloce né troppo lento, e non lo fa come uno che se ne
vuole andare».

«Egli indossa la sua veste non troppo su né troppo giù sul corpo, non troppo
stretta né troppo lenta sul corpo, né il vento gli fa sventolare via la veste
dal corpo. La polvere e la sporcizia non contaminano il suo corpo».

«Quando egli va nella foresta, egli siede a terra o in un posto già pronto. Dopo
essersi seduto, si lava i piedi. Non si preoccupa di prendersi cura dei suoi
piedi. Dopo essersi lavato i piedi, si siede a gambe incrociate, erige il suo
corpo e fissa la consapevolezza davanti a lui. Egli non occupa la sua mente con
afflizioni proprie o con le afflizioni degli altri o con le afflizioni di
entrambi. Egli siede con la mente intenta al benessere proprio, al benessere
degli altri e al benessere di entrambi, nei fatti al benessere di tutto il
mondo».

«Quando va in monastero, egli insegna il Dhamma all’uditorio. Egli non lusinga
né rimprovera chi ascolta, egli istruisce, esorta, risveglia e incoraggia
l’uditorio con soli discorsi di Dhamma. Il discorso che esce dalle sue labbra ha
otto qualità: è distinto, comprensibile, melodioso, ascoltabile, risuonante,
incisivo, profondo e sonoro, ma mentre la sua voce può essere udita fino ai
confini dell’uditorio, essa non si estende al di là di tale stesso uditorio.
Quando le persone sono state istruite, esortate, risvegliate e incoraggiate da
lui, loro si alzano dal luogo in cui siedono e vanno via guardando solo verso di
lui, senza occuparsi di nient’altro».

«Signore, abbiamo visto il Maestro Gotama camminare, lo abbiamo visto stare in
piedi, lo abbiamo visto all’interno stare seduto in silenzio, lo abbiamo visto
all’interno mangiare, lo abbiamo visto all’interno stare seduto in silenzio dopo
aver mangiato, lo abbiamo visto impartire la benedizione dopo aver mangiato, lo
abbiamo visto andare in monastero, lo abbiamo visto stare seduto in monastero in
silenzio, lo abbiamo visto in monastero mentre insegnava il Dhamma a un
uditorio. Questo è il Maestro Gotama. Questo egli è, e pure di più».

Quando ciò fu detto, Brahmāyu il brāhmaṇa si alzò dal luogo in cui sedeva e,
sistemandosi la veste superiore su una spalla, levò le mani giunte verso il
luogo in cui si trovava il Beato ed esclamò queste parole per tre volte: «Onore
al Beato, realizzato e completamente illuminato! Onore al Beato, realizzato e
completamente illuminato! Onore al Beato, realizzato e completamente illuminato!
Auguriamoci di incontrare qualche volta il Maestro Gotama. Auguriamoci di
conversare insieme».

\suttaRef{M. 91}

Una volta il Beato viveva a Campā, sulla riva del lago Gaggarā. Allora, a
mezzogiorno il capofamiglia Vajjiyamāhita uscì da Campā per incontrare il Beato.
Per strada, però, pensò: «Non è ancora il momento per incontrare il Beato, egli
è in ritiro. E non è ancora il momento per vedere i bhikkhu che praticano la
meditazione, loro sono in ritiro. E se io mi recassi al parco che appartiene
alle altre sette?».

Là si recò. In quel momento gli asceti itineranti di altre sette si erano
riuniti, ed erano seduti a parlare di ogni genere di bassi discorsi, urlando e
facendo un fragoroso e rumoroso clamore. Quando videro il capofamiglia
Vajjiyamāhita che da lontano si avvicinava, si acquietarono gli uni con gli
altri, dicendo: «Signori, che non si faccia rumore qui. Non fate rumore. Il
capofamiglia Vajjiyamāhita sta arrivando ed egli è un seguace del monaco Gotama.
Se a Campā vivono dei laici vestiti di bianco che sono seguaci del monaco
Gotama, lui è uno di loro. Queste rispettabili persone amano poco rumore e sono
addestrate a farne poco, e raccomandano di fare poco rumore. Se forse egli vede
che noi siamo una congregazione poco dedita al rumore, penserà che valga la pena
di avvicinarsi».

Allora gli asceti itineranti rimasero in silenzio. Il capofamiglia Vajjiyamāhita
andò da loro e scambiò saluti. Poi si mise a sedere da un lato. Loro gli
chiesero: «Capofamiglia, è vero, come sembra, che il monaco Gotama disapprova
l’austerità e condanna e censura senza alcuna eccezione tutti coloro che
conducono la dura vita dell’austerità?».

«Non è così, signori. Il Beato disapprova quel che dev’essere disapprovato e
raccomanda quel che dev’essere raccomandato. Nel farlo, però, egli è uno che
parla con discernimento, non è uno che conduce affermazioni unilaterali».

Allora un asceta itinerante gli disse: «Aspetta un attimo, capofamiglia, questo
monaco Gotama che tu lodi è un nichilista (uno che porta via): egli non descrive
nulla, in coerenza con quanto tu hai detto di lui».

«Al contrario, signori, dico a ragione ai venerabili che il Beato ha descritto
come certe cose sono salutari e come certe altre sono non salutari. Così, egli è
perciò uno che descrive qualcosa, non è uno che non lo fa».

Quando ciò fu detto, gli asceti itineranti rimasero in silenzio.

\suttaRef{A. 10:94}

\narrator{Secondo narratore.} Saccaka, un figlio di Nigaṇṭha, venne a disputare
con il Buddha a Vesālī. Il Buddha descrive come il suo sforzo precedente
l’Illuminazione gli fece scoprire che la mortificazione non conduce da nessuna
parte. Egli disse:

\voice{Prima voce.} «Ho fatto esperienza dell’insegnamento del Dhamma a un
uditorio di molte centinaia di persone. Forse qualcuno ha fantasticato: “Il
monaco Gotama sta predicando il Dhamma per me personalmente”. Ma la cosa non
dovrebbe essere considerata in questo modo. Un Perfetto espone il Dhamma agli
altri per offrire loro la conoscenza. Quando il discorso è terminato, allora io
consolido la mia mente in me stesso, la acquieto, la conduco all’unificazione e
la concentro sullo stesso oggetto di consapevolezza sulla quale la stavo
concentrando in precedenza».

«Così ci si attende da lui, visto che il Maestro Gotama è realizzato e
completamente illuminato. Il Maestro Gotama ha, però, mai dormito di giorno?».

«Durante l’ultimo mese della stagione calda, tornando dal giro per la questua
dopo il pasto, ho sperimentato di deporre la mia veste superiore fatta di toppe
piegata in quattro, di giacere sul lato destro e di addormentarmi consapevole e
in piena presenza mentale».

«Alcuni monaci e brāhmaṇa dicono che si tratta del modo di dimorare di un uomo
preda dell’illusione».

«Non è in quel modo che un uomo è preda dell’illusione o non è preda
dell’illusione. Io chiamo preda dell’illusione colui nel quale le contaminazioni
che inquinano, che rinnovano l’esistenza, maturano in futura sofferenza e
conducono alla nascita, all’invecchiamento e alla morte, non sono abbandonate.
Perché è con il non abbandono delle contaminazioni che un uomo è preda
dell’illusione. Io chiamo non preda dell’illusione colui nel quale queste
contaminazioni sono abbandonate. Perché è con l’abbandono delle contaminazioni
che un uomo non è preda dell’illusione. Proprio come una palma non può più
crescere quando la sua corona è tagliata, così pure in un Perfetto queste
contaminazioni sono abbandonate, eliminate, recise alla radice, rese come un
ceppo di palma, abolite e non più soggette a sorgere in futuro».

Quando ciò fu detto, Saccaka osservò: «È meraviglioso, Maestro Gotama, è
magnifico come, quando il maestro Gotama è attaccato in continuazione con
osservazioni personali, il colore della sua pelle risplende, il colore del suo
volto schiarisce, come avviene in chi è realizzato e completamente illuminato!
Ho avuto esperienza di entrare in discussione con Pūraṇa Kassapa, ed egli mi
prevaricò e deviò il discorso e mostrò perfino rabbia, odio e scontrosità. Lo
stesso avvenne con Makkhali Gosāla e con altri. E ora, Maestro Gotama, noi
andiamo. Siamo impegnati e abbiamo molto da fare».

\suttaRef{M. 36}

\narrator{Secondo narratore.} Tuttavia Saccaka non si convinse e conservò i
propri punti di vista.

\narrator{Primo narratore.} C’è un episodio che mostra come il Buddha non fosse
immune dalle malattie.

\voice{Prima voce.} Una volta il Beato soggiornava nel Parco di Nigrodha a
Kapilavatthu, nel territorio dei Sakya. Era appena guarito da una malattia.
Allora Mahānāma il Sakya andò da lui e disse: «Signore, da lungo tempo conosco
il Dhamma insegnato dal Beato in questo modo: “La conoscenza è per chi è
concentrato, non per chi non è concentrato”. Viene prima la concentrazione,
Signore, e poi la conoscenza, o prima la conoscenza e poi la concentrazione?».

Il venerabile Ānanda pensò: «Il Beato si è appena rimesso da una malattia, e
questo Sakya Mahānāma gli rivolge una domanda davvero profonda. E se io
prendessi Mahānāma da parte e gli insegnassi il Dhamma?».

Così fece, e gli disse: «Il Beato ha dichiarato la virtù dell’allievo,
concentrazione e comprensione, e ha dichiarato la virtù dell’adepto,
concentrazione e comprensione. La virtù dell’allievo è quella di un bhikkhu
virtuoso che, contenuto con il contenimento del \emph{Pātimokkha}, perfetto
nella condotta e nel modo di vivere, teme il più piccolo errore, si addestra
portando a effetto i precetti della virtù. La sua concentrazione è quella di un
bhikkhu che entra e dimora in uno dei quattro jhāna. La sua comprensione e
quella di un bhikkhu che comprende quel che in realtà è: “Questa è la
sofferenza, questa è l’origine della sofferenza, questa è la cessazione della
sofferenza, questo è il Sentiero che conduce alla cessazione della sofferenza”.
Ora, nel caso dell’adepto, il nobile discepolo che già possiede questa virtù,
concentrazione e comprensione, mediante realizzazione di se stesso qui e ora,
entra e dimora nella liberazione della mente e nella liberazione della mente
mediante comprensione\footnote{Oppure Liberazione mediante saggezza
  (\emph{paññā-vimutti}) (Nyp.).} immacolata per l’esaurimento delle
contaminazioni».

\suttaRef{A. 3:73}

\narrator{Primo narratore.} Il Buddha era di statura normale. Lo si può supporre
dalla storia del suo cambio di veste con l’Anziano Mahā-Kassapa, che sarà
offerta in seguito, e dal seguente episodio.

\voice{Seconda voce.} Avvenne questo. Il Beato stava soggiornando a Sāvatthī nel
Boschetto di Jeta, nel parco di Anāthapiṇḍika, e a quel tempo il venerabile
Nanda, il figlio della zia del Beato, si trovava là. Egli era di bell’aspetto, e
ispirava fiducia e sicurezza. Era quattro dita più basso del Beato. Era solito
indossare una veste della stessa misura della veste del Sublime e, quando i
bhikkhu più anziani videro il venerabile Nanda che arrivava da lontano, lo
scambiarono per il Beato e, perciò, si alzarono dal luogo in cui sedevano.
Quando egli però arrivò, si accorsero del loro errore. Disapprovarono,
mormorarono e protestarono: «Come può il venerabile Nanda indossare una veste
della stessa misura della veste del Sublime?».

Lo raccontarono al Beato. Egli rimproverò il venerabile Nanda, e istituì questa
regola d’addestramento: «Qualsiasi bhikkhu che indossi una veste della stessa
misura della veste del Sublime commette un’infrazione che comporta espiazione.
Le misure della veste del Sublime sono: nove spanne di lunghezza e sei spanne di
larghezza, della spanna del Sublime».

\suttaRef{Vin. Sv. Pāc. 92}

\narrator{Primo narratore.} La storia dell’Anziano Vakkali è qui opportuna in
quanto illustra l’attitudine del Buddha a essere presente personalmente.

\voice{Prima voce.} Così ho udito. Una volta, quando il Beato soggiornava a
Rājagaha, nel Boschetto di Bambù, nel Sacrario degli Scoiattoli, il venerabile
Vakkali viveva nella casa di un vasaio. Era afflitto, sofferente e gravemente
malato. Egli disse ai suoi monaci attendenti: «Amici, andate dal Beato, prestate
omaggio a lui da parte mia, con il vostro capo ai suoi piedi, e dite: “Signore,
il bhikkhu Vakkali è afflitto, sofferente e gravemente malato. Egli presta
omaggio con il suo capo ai piedi del Beato”. Poi dite questo: “Signore, sarebbe
bene che il Beato andasse dal bhikkhu Vakkali mosso da compassione”».

«Sì, amico», risposero i bhikkhu. Andarono dal Beato e gli portarono il
messaggio e la richiesta. Il Beato acconsentì in silenzio. Poi si vestì, prese
la ciotola e la veste superiore, e si recò dal venerabile Vakkali. Il venerabile
Vakkali lo vide arrivare e cercò di alzarsi dal letto. Il Beato disse: «Va bene
così, Vakkali. Non alzarti dal letto. Ci sono posti a sedere preparati, mi
metterò a sedere qui». Egli si mise a sedere in uno dei posti preparati. Poi
disse: «Spero che le cose ti vadano bene, Vakkali, spero che tu ti senta a tuo
agio, che i tuoi dolori stiano andando via, che non stiano aumentando, che
sembrino diminuire, non aumentare».

«Signore, le cose non vanno bene per me. Non mi sento a mio agio. I miei dolori
stanno crescendo, non andando via, sembrano aumentare, non diminuire».

«Spero che tu non abbia preoccupazioni e rimorsi, Vakkali».

«Certamente, Signore, non ho alcuna preoccupazione né rimorsi».

«Spero, allora, che tu non abbia nulla da rimproverarti a riguardo del
comportamento virtuoso».

«Non ho nulla da rimproverarmi a riguardo del comportamento virtuoso, Signore».

«Se non hai nulla da rimproverarti, Vakkali, per che cosa ti preoccupi e provi
rimorso?».

«Signore, da lungo tempo desideravo venire a vedere il Beato, ma non ho avuto
abbastanza forza fisica per farlo».

«Va bene così, Vakkali. Perché vuoi vedere questo corpo immondo? Colui che vede
il Dhamma vede me, e quando vede me vede il Dhamma. Cosa ne pensi, Vakkali, la
forma materiale è permanente o impermanente?».

\narrator{Secondo narratore.} Il Buddha proseguì ripetendo il discorso che aveva
offerto ai bhikkhu del gruppo dei cinque dopo l’Illuminazione.

\voice{Prima voce.} Il Beato, dopo aver impartito al venerabile Vakkali questa
istruzione, si alzò dal posto in cui sedeva e andò al Picco dell’Avvoltoio.

Subito dopo che se ne fu andato, il venerabile Vakkali disse ai suoi monaci
attendenti: «Venite, amici, mettetemi su una lettiga e portatemi al Picco Nero
sulle pendici di Isigili. Come può uno come me pensare di morire in una casa?».

«Sì, amico», risposero, e fecero come aveva detto.

Il Beato trascorse il resto di quella giornata e di quella notte sul Picco
dell’Avvoltoio. Quando la notte fu terminata, si rivolse ai bhikkhu in questo
modo: «Venite, bhikkhu, andate dal bhikkhu Vakkali e ditegli così: “Amico
Vakkali, ascolta che cosa le divinità hanno detto al Beato. La notte scorsa due
divinità dall’aspetto meraviglioso, che illuminavano tutto il Picco
dell’Avvoltoio, si sono recate dal Beato e, dopo avergli prestato omaggio, una
di loro ha detto: ‘Signore, il bhikkhu Vakkali ha predisposto il suo cuore alla
Liberazione’. E l’altra divinità ha detto: ‘Signore, egli otterrà certamente la
completa Liberazione’. E il Beato questo ti dice, amico: ‘Non avere paura,
Vakkali, non avere paura. La tua morte sarà innocente da malvagità, il
compimento del tuo tempo sarà innocente da malvagità’ ”».

«E sia, Signore», risposero. Poi andarono dal venerabile Vakkali e gli dissero:
«Amico, ascolta un messaggio del Beato e di due divinità».

Il venerabile Vakkali disse ai suoi monaci attendenti: «Venite, amici, fatemi
scendere dal letto, com’è possibile per uno come me ascoltare il messaggio del
Beato stando seduto su di un seggio alto?».

«Sì, amico», risposero, e fecero come aveva detto. Poi gli fu comunicato il
messaggio.

Egli disse: «Ora amici, prestate omaggio al Beato da parte mia, con il vostro
capo ai suoi piedi, e dite: “Signore, il bhikkhu Vakkali è afflitto, sofferente
e gravemente malato. Egli presta omaggio con il suo capo ai piedi del Beato, e
dice questo: ‘Signore, non ho dubbi che la forma materiale, la sensazione, la
percezione, le formazioni [mentali] e la coscienza sono impermanenti. Non ho
incertezze in relazione al fatto che quello che è impermanente è sofferenza. Non
ho desiderio né brama né affezione per quello che è impermanente, doloroso e
soggetto al cambiamento, in relazione a questo non ho incertezze’ ”».

«Sì, amico», risposero. Poi andarono. Subito dopo che se ne furono andati il
venerabile Vakkali si tolse la vita.

Quando i bhikkhu furono andati dal Beato e gli riferirono le parole del
venerabile Vakkali, Egli disse: «Andiamo al Picco Nero sulle pendici di Isigili,
bhikkhu, dove l’uomo di rango Vakkali si è tolto la vita».

«E sia, Signore», risposero. Allora il Beato andò al Picco Nero sulle pendici di
Isigili con un gruppo di bhikkhu. Egli vide da lontano il corpo privo di sensi
del venerabile Vakkali che giaceva su di un letto. Nello stesso tempo, però, una
nebbia fumosa, un’ombra cupa si muoveva verso est e verso ovest, e verso nord e
verso sud, come pure verso tutte le direzioni intermedie. Allora il Beato disse
ai bhikkhu: «Bhikkhu, vedete quella nebbia fumosa, quell’ombra cupa?».

«Sì, Signore».

«Bhikkhu, è Māra il Malvagio. Sta cercando la coscienza dell’uomo di rango
Vakkali: “Dove s’è stabilita la coscienza dell’uomo di rango Vakkali?”. L’uomo
di rango Vakkali, però, bhikkhu, ha ottenuto il Nibbāna definitivo, senza che la
sua coscienza si sia stabilita da una qualche parte».

\suttaRef{S. 22:87}

\narrator{Primo narratore.} Nei Piṭaka sono riportati vari esempi di bhikkhu che
si tolgono la vita. Il Buddha disse che ciò non era riprovevole a una sola
condizione: che il bhikkhu fosse già un Arahant, privo di brama, odio o
illusione, o che lo fosse diventato prima di morire, e che il togliersi la vita
fosse connesso alla sola ragione di porre fine a una malattia incurabile.
Altrimenti, togliere la vita a un essere umano, o consigliargli la morte,
rappresenta una delle quattro Sconfitte, o infrazioni capitali, che comportano
la permanente espulsione dal Saṅgha – le altre tre sono il furto, il rapporto
sessuale, e affermare il falso in relazione a conquiste spirituali – benché il
tentato suicidio sia un’infrazione minore di atto errato.

\narrator{Secondo narratore.} Si è in precedenza riferito come il Buddha
menzionò i sei Buddha che lo avevano preceduto. Egli menzionò pure il Buddha che
gli sarebbe succeduto in futuro, quel che sarebbe avvenuto dopo al suo stesso
insegnamento e dopo che il suo ricordo sarebbe del tutto svanito dal mondo.

\voice{Prima voce.} «Quando la vita degli esseri umani aumenterà a ottantamila
anni, il beato Metteyya, realizzato e completamente illuminato, sorgerà nel
mondo, perfetto nella conoscenza e nella condotta, sublime, conoscitore dei
mondi, incomparabile guida degli uomini che devono essere addestrati, insegnante
di dèi e uomini, illuminato, beato, proprio come ora lo sono io. Egli realizzerà
se stesso mediante conoscenza diretta, e lo dichiarerà a questo mondo con i suoi
deva, con i suoi Māra e con le sue divinità, in questa generazione con i suoi
monaci e brāhmaṇa, con i suoi principi e uomini, proprio come ora ho fatto io.
Insegnerà il Dhamma che è salutare al principio, salutare nel mezzo e salutare
alla fine, con il significato e il senso letterale, e spiegherà la santa vita
che è assolutamente perfetta e pura, proprio come ora ho fatto io».

\suttaRef{D. 26}

Questo fu detto dal Beato, dal Realizzato, così ho udito: «Bhikkhu, io sono un
brāhmaṇa, abituato alla liberalità e munifico. Questo è il mio ultimo corpo. Io
sono il medico supremo. Voi siete i figli del mio petto, nati dalle mie labbra,
nati dal Dhamma, eredi del Dhamma, non di cose materiali. Ci sono due tipi di
doni: il dono delle cose materiali e il dono del Dhamma. Il più grande di questi
è il dono del Dhamma».

\suttaRef{Iti. 100}

\label{pag222}%
«Ora, bhikkhu, se gli altri dovessero chiedere a un bhikkhu: “Quali sono le
prove e le certezze in ragione delle quali, tu, venerabile signore, dici: ‘Il
Beato è completamente illuminato, il Dhamma è ben proclamato, il Saṅgha è sulla
buona strada?’ ”. Allora, per rispondere rettamente, dovete rispondere così:
“Ecco, amici, mi sono avvicinato al Beato per ascoltare il Dhamma. Il Maestro mi
ha mostrato il Dhamma in ogni stadio, sempre più in alto, per ogni livello
superiore, in tutti i suoi aspetti. In accordo con questo suo comportamento,
giungendo a una conoscenza diretta di un certo insegnamento (per l’esattezza,
uno dei quattro stadi del Sentiero della Realizzazione) tra gli insegnamenti
insegnati nel Dhamma, io ho raggiunto il mio scopo. Allora ebbi fiducia nel
Maestro in questo modo: ‘Il Beato è completamente illuminato, il Dhamma è ben
proclamato, il Saṅgha è sulla buona strada’ ”. Quando la fede di qualcuno nel
Perfetto è impiantata e radicata con queste prove, queste frasi e queste
sillabe, allora la sua fede la si dice supportata dall’evidenza, radicata nella
visione, nel suono e invincibile [se avversata] da un monaco, da un brāhmaṇa, da
Māra, da Brahmā o da chiunque altro nel mondo».

\suttaRef{M. 47}

«Quando i discepoli del Maestro Gotama sono consigliati e istruiti da Lui,
conseguono il supremo scopo del Nibbāna, o qualcuno non lo consegue?».

«Qualcuno lo consegue, brāhmaṇa, qualcun altro no».

«Perché succede questo, Maestro Gotama, dal momento che il Nibbāna c’è, e anche
il Sentiero che conduce a esso c’è, e la guida è il Maestro Gotama?».

«Per quanto concerne tutto questo, brāhmaṇa, io, di rimando, ti porrò una
domanda. Rispondi a essa come preferisci. Cosa ne pensi: ti sono famigliari le
strade che conducono a Rājagaha?».

«Sì, Maestro Gotama, mi sono famigliari».

«Cosa ne pensi: supponiamo che ci sia un uomo che vuole andare a Rājagaha, che
ti si avvicini e ti dica: “Signore, indicami la strada per Rājagaha”. E che tu
gli risponda: “Ora, buon uomo, questa strada va a Rājagaha. Seguila per un po’ e
vedrai un tal villaggio, poi una tal città, e infine Rājagaha con i suoi
giardini, boschetti, campagne e laghi”. Benché così consigliato e istruito da
te, che egli invece prenda una strada sbagliata e prosegua verso occidente. E
poi che arrivi un secondo uomo e, dopo averti rivolto la stessa domanda e
ricevuto da te lo stesso consiglio e la stessa istruzione, egli giunga senza
problemi a Rājagaha. Ora, dal momento che Rājagaha c’è, e anche il sentiero che
conduce a essa c’è, e la guida sei tu stesso, perché succede che un uomo prenda
la strada sbagliata e vada verso occidente e un altro uomo giunga senza problemi
a Rājagaha?».

«Che cosa ho io a che fare con tutto questo, Maestro Gotama? Io sono solo colui
che indica la via».

«Così, brāhmaṇa, allo stesso modo il Nibbāna c’è, e anche il Sentiero che
conduce a esso c’è, e la guida sono io stesso, tuttavia quando i miei discepoli
sono consigliati e istruiti da me, alcuni ottengono il Nibbāna e altri no. Che
cosa ho io a che fare con tutto questo, brāhmaṇa? Un Perfetto è solo colui che
indica la via».

\suttaRef{M. 107 (condensato)}

Una volta alcuni asceti itineranti di altre sette andarono dal venerabile
Anurādha e gli chiesero: «Amico Anurādha, chi è Perfetto, il sommo tra gli
uomini, il supremo tra gli uomini, uno che ha conseguito la realizzazione
suprema, quando viene descritto da un altro Perfetto, in quale dei quattro
seguenti modi viene descritto? Dopo la morte un Perfetto esiste. Oppure, dopo la
morte un Perfetto non esiste. Oppure, dopo la morte un Perfetto sia esiste sia
non esiste. Oppure, dopo la morte un Perfetto né esiste né non
esiste».\pagenote{%
  Si tratta di quattro delle “dieci cose non dichiarate” (cf.~cap.~12,
  pag.~\pageref{pag230} -- \emph{Una volta, quando il Beato era entrato a
    Rājagaha\ldots}), le quali tutte implicano un’affermazione,
  indipendentemente dal fatto che la risposta sia sì o no. I Greci erano soliti
  chiedere: «Usi un bastone per picchiare tua moglie?», e sia che la risposta
  fosse “sì” sia “no”, la conclusione era: «Allora tu picchi tua moglie». Per le
  ragioni per cui il Buddha rifiutò di rispondere si veda la fine di questo
  capitolo.}

«Amici, un Perfetto, descrivendolo, non lo descrive in uno di questi quattro
modi».

Quando ciò fu detto, loro rimarcarono: «Costui sarà un nuovo bhikkhu oppure un
Anziano che non da molto ha abbracciato la vita religiosa, e che è stolto e
privo d’esperienza». Poi, privi di fiducia nel venerabile Anurādha e pensando
che egli avesse da poco abbracciato la vita religiosa, si alzarono dal luogo in
cui erano seduti e se ne andarono. Poi, appena se ne furono andati, egli si
chiese: «Se mi avessero rivolto altre domande, come avrei potuto rispondere in
modo da dire quel che il Beato dice senza travisarlo con ciò che nei fatti non
è, ed esprimendo idee in accordo con il Dhamma, senza che nulla sia
legittimamente deducibile dalle mie affermazioni e che possa offrire ragioni per
incolparmi?». Così si recò dal Beato e gli raccontò quanto era avvenuto.

«Cosa ne pensi, Anurādha, la forma materiale è permanente o impermanente?».

«Impermanente, Signore».

\narrator{Secondo narratore.} Il Buddha proseguì come aveva fatto nel Secondo
Sermone pronunciato ai bhikkhu del gruppo dei cinque, e dopo chiese:

«Cosa ne pensi, Anurādha? Pensi che la forma materiale sia il Perfetto?».

«No, Signore».

«Pensi che la sensazione … la percezione … le formazioni [mentali] … la
coscienza sia il Perfetto?».

«No, Signore».

«Cosa ne pensi, Anurādha? Pensi che il Perfetto sia nella forma materiale?».

«No, Signore».

«Pensi che il Perfetto sia separato dalla forma materiale?».

«No, Signore».

«Pensi che il Perfetto sia nella sensazione … sia separato dalla sensazione …
sia nella percezione … sia separato dalla percezione … sia nelle formazioni
[mentali] … sia separato dalle formazioni [mentali] … sia nella coscienza … sia
separato dalla coscienza?».

«No, Signore».

«Cosa ne pensi, Anurādha? Pensi che il Perfetto sia la forma materiale, la
sensazione, la percezione, le formazioni [mentali] e la coscienza?».

«No, Signore».

«Cosa ne pensi, Anurādha? Pensi che il Perfetto sia privo di forma materiale,
privo di sensazione, privo di percezione, privo di formazioni [mentali], privo
di coscienza?».

«No, Signore».

«Anurādha, quando un Perfetto è davanti a te qui e ora, incomprensibile come
vero e fondato, è appropriato dire di lui: “Amici, chi è Perfetto, il sommo tra
gli uomini, il supremo tra gli uomini, uno che ha conseguito la realizzazione
suprema, quando un Perfetto lo descrive, non lo descrive in uno dei quattro
seguenti modi? Dopo la morte un Perfetto esiste. Oppure, dopo la morte un
Perfetto non esiste. Oppure, dopo la morte un Perfetto sia esiste sia non
esiste. Oppure, dopo la morte un Perfetto né esiste né non esiste”?».

«No, Signore».

«Bene, Anurādha, bene. Quel che io descrivo, ora come prima, è la sofferenza e
la cessazione della sofferenza».

\suttaRef{S. 44:2}

«Perché il Beato non ha risposto a queste domande? Perché esse descrivono tutte
un Perfetto dopo la morte nei termini di forma (e così via)» (S. 44:3). «Perché
sono state poste da chi non è libero dal desiderio, dall’amore, dalla sete,
dalle febbre e dalla bramosia per la forma (e così via)» (S. 44:5). «Perché sono
state poste da chi è attratto dalla forma (e così via) e anche dall’esistenza e
dall’attaccamento e dalla brama, e non sa come queste cose giungano a
cessazione» (S. 44:6). «Queste domande fanno parte della boscaglia delle
opinioni … della catena delle opinioni: sono collegate alla sofferenza,
all’angoscia, alla disperazione e alla febbre, e non conducono al distacco, al
disincanto, alla cessazione, all’acquietamento, alla conoscenza diretta,
all’Illuminazione, al Nibbāna».

\suttaRef{M. 72}

\label{pag226}%
«Colui che è Così-Andato (Tathāgata, un Perfetto)\pagenote{%
  La parola \emph{tathāgata} (qui tradotta non letteralmente con “il Perfetto”)
  fu inizialmente usata dal Buddha per se stesso subito dopo l’Illuminazione
  (cap.~3, pag.~\pageref{pag41} -- \emph{In quell’occasione due mercanti,
    Tapussa e Bhalluka, stavano viaggiando\ldots}). In seguito la utilizzò per
  gli Arahant. Il Commentario la fa derivare in vari modi (ne tratta in sette
  pagine): «perché Egli è \emph{tathāgato}, così-venuto, per mezzo
  dell’aspirazione all’Illuminazione, come fecero i precedenti Buddha; perché
  Egli è \emph{tathāgato}, così-andato, per mezzo della pratica e della
  realizzazione, come pure i precedenti Buddha; perché Egli è
  \emph{tatha-lakkhaṇaṃ āgato}, venuto a conoscenza della caratteristica della
  realtà», ecc.}
è qui e ora inconoscibile, dico. Nel dire questo, nel proclamare questo, sono
stato senza alcun fondamento, vanamente, falsamente, erratamente frainteso da
alcuni monaci e brāhmaṇa in questo modo: “Il monaco Gotama è uno che porta via
(verso il nichilismo) perché egli descrive l’annullamento, la perdita, la
non-esistenza di una creatura esistente”».

\suttaRef{M. 22}

«Il sé può essere acquisito in questi tre modi. Il sé grossolano, il sé
costituito dalla mente e il sé privo di forma … Il primo ha una forma
(materiale), consiste di quattro grandi elementi e consuma cibo fisico. Il
secondo è costituito dalla mente, è completo di tutte le sue parti, non mancante
di alcuna facoltà. Il terzo è privo di forma e consiste nella percezione … Io
insegno il Dhamma per l’abbandono delle acquisizioni del sé affinché in voi, che
mettete l’insegnamento in pratica, possano essere abbandonate le qualità
contaminate e accresciute quelle purificatrici, e affinché voi possiate,
realizzando voi stessi qui e ora con la conoscenza diretta, entrare e dimorare
nella pienezza della perfezione conoscitiva … Se si pensa che ciò sia un
dimorare doloroso, non è così. Al contrario, così facendo c’è contentezza,
felicità, tranquillità, consapevolezza, piena presenza mentale e un piacevole
dimorare».

\narrator{Secondo narratore.} Il Buddha continuò a dire che, da una rinascita
all’altra, a ognuno di questi tre modi di acquisizione del sé può seguirne un
altro. Stando così le cose, non è possibile sostenere a ragione che solo uno di
essi è vero e che gli altri sono errati. Si può solo dire che il termine che
descrive ognuno di essi non è adatto agli altri due. Proprio come il latte da
una mucca, la cagliata dal latte, il burro dalla cagliata, il burro chiarificato
dal burro, l’estratto di burro chiarificato dal burro chiarificato, ogni termine
è adatto a ciò che descrive e a nessuno degli altri, benché, tuttavia, ognuno
non sia slegato dall’altro. Il Buddha concluse:

\voice{Prima voce.} «Questi sono usi del mondo, linguaggio del mondo, termini
per la comunicazione del mondo, descrizioni del mondo, tramite i quali un
Perfetto comunica senza fraintenderli».

\suttaRef{D. 9 (condensato)}

